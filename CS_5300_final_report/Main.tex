\documentclass{report}
\usepackage{mathtools}
\usepackage{amssymb}
\usepackage{blindtext}
\usepackage{hyperref}
\usepackage[table]{xcolor}
\usepackage{tikz}
\usepackage{listings}
\usetikzlibrary{arrows}
\usetikzlibrary{positioning}
\begin{document}
	\textbf{Abstraction}\\
		Hello, World!!\\
	\textbf{Introduction}\\
	Nowadays, Machine Learning becomes more and more popular. After the success of AlphaGo, the field catches the attentions all over the world.  In 1959, Arthur Samuel defined machine
	learning as a ``Field of study that gives computers the ability to learn without being explicitly programmed.''[] Developing traditional AI application without using machine learning
	strategy is the process that transfers human knowledge to formal computer algorithm and implements it in programming languages. This approach gives the computer the knowledge needed
	to complete the tasks rather than the abilities to learn the knowledge. Traditional AI approach without machine learning solves the Game Play problem typically in the following steps.
	\begin{enumerate}
		\item Given the current game state $S_0$, calculate possible next game states $S_{next} = \{s_i|s_0\xrightarrow[]{move_i}s_i\}$

		\item Evaluate all game states in $S_{next}$ with the Evaluation Function f. Find the best next game state $S_{best}$ where $f(s_{best}) = max\{f(s_i)\,|\,s_i \epsilon s_{next}\}$ and corresponding
			move $m_{best}$. ($s_0 \xrightarrow[]{move_{best}} s_{best}$)
		\item Apply $move_{best}$.
	\end{enumerate}
	Obviously the key element in Game Play AI application is the Evaluation Function.  Designing Evaluation Function is the process to transfer human knowledge or heuristics into a formula
	that gives a meaningful Evaluation Score to help with decision making.\\
	The problem with the traditional AI
	\begin{enumerate}
		\item The Evaluation Function is hard to design. There are too many factors needed to be considered in order to give a meaningful score. 
		\item Typically the Evaluation Function only considers the current game state without looking at the future, which makes the Evaluation Score no so meaningful.  Min-Max search[]
		was introduced to solve the problem. But as the number of game states grows exponentially, this approach becomes not practical.
	\end{enumerate}
	Machine Learning overcomes the problem by giving computer the abilities to learning how to evaluate the game states. We use Artificial Neural Network as the model of Machine Learning. A
	game state is converted to a input vector which gets plugged into the Neural Network. The output of the Network is the Evaluation Score of the game state. We use back-propagation algorithm
	as our learning algorithm. In the next section, we will talk about our approach in details.\\

	\textbf{Backpropagartion Details}\\
	\blindtext\\
	
	\textbf{Implementation Details}\\
	When we were assigned the task of writing a blokus-learning algorithm, our first response was to break our task into pieces:
	\begin{itemize}
		\item Some sort of host-program, game-engine, something that could manage the game board, keep track of it's current state, draw it on the screen, keep track of the players' pieces,
			and calculate possible moves for the players.
		\item A function that would `act' for the player, it would somehow take in the current board, and decide what piece to place. This function would be called by the game-engine each
			turn.
		\item Some kind of ``learning''-method, it would hopefully change the way the `act'ing-function would operate.
	\end{itemize}
	Before we consided the exact implementation of these pieces, we knew that we could make our lives much easier if we didn't start of scratch, and instead build off of other developers's code.
	That's when we discovered a repository on github that already had developed the game-engine, players, and simple user-interface.\\
	It out-of-the box ran the user-player aganist a random-computer-player, and drew the board on an ASCII grid printed out on the terminal. We thought this was a perfect starting point,
	so we began exploring through the code, to find develop	a basic idea of the flow of the program, and to start getting an idea of how we can change this program to suit our needs.
	Each player was an object containing an array of pieces, a score, a name, corner to start from, and a function pointer. Each player was given a function-pointer to call when it needed
	to decide what piece to place where (the `act'-ing function). The player objects are handed their `act'ing-function pointer on construction. Originally, the program had four `act'ing-functions
	defined: one that read from standard input from user as	a human player, another that randomly placed pieces, MinMax algorithm, and a Greedy algoritm. By changing the function pointers of
	the objects, we could easisly remove the human-player, and have two computer-players compete.\\
	Now that we knew we had a soild base, we started to consider in more detail the implementation of our ``learning'' `act'ing-function. Even though theorically any ``learning'' function would do,
	we felt we would have the best results with a typical netural net desgin. Then came the more difficult part were we had to decide what should be a part of the input layer and output layer,
	and how it should be represented. At first we considered somehow inputting the current board, and the availible pieces, and then having the neural net output which piece, x-y cordinate,
	and rotation for that piece. That idea quickly fell apart with how to represent a variable-length list of shapes, and how to train rotation and flips, as such operations aren't linearly
	separable. The second idea was to input a future possible board, and have the neural net output a score on how well it thinks it is doing in the game. Such input the host-program can easily
	compute, and a output it can work with to make the best move, simply by picking the highest-scored future move. With this particular output of the neural net, training is very easy. To
	improve the neural net. we can compare the output of the neural net to the final score of the player at the end of the game. The representation of the future board can be given as a large set of
	integers, being that the blokus board is a set of small squares, which can only have 5 possible states. (1 for Player1, 2 for Player2, ..., 0 for empty space) This makes the `act'-ing
	function for our ``learning'' players equailivent to:
	\lstset{language=Python}
	\lstset{basicstyle=\footnotesize}
	\begin{lstlisting}
possible_moves = getFutureMoves(availible_pieces)
possible_moves_scores = []
for move in possible_moves:
	future_board = current_board.copy()
	future_board.playMove(move);
	input_vector = [];
	for x in range(0, future_board.width):
		for y in range(0, future_board.height):
			cellvalue = 0;
			if(future_board.WhoOwns(x, y) == player1):
				cellvalue = 1;
			if(future_board.WhoOwns(x, y) == player2):
				cellvalue = 2;
			if(future_board.WhoOwns(x, y) == player3):
				cellvalue = 3;
			if(future_board.WhoOwns(x, y) == player4):
				cellvalue = 4;
			input_vector.append(cellvalue);
	score = evaulate_neural_network(weights, input_vector)
	possible_moves_scores.append(score)
highest_score_index = # highest value's index in possible_moves_scores #
return possible_moves[highest_score_index]
	\end{lstlisting}
	
	\textbf{Experimentation/Evaulation}\\
	\blindtext\\
	
\end{document}


















